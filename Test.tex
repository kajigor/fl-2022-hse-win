\documentclass[12pt]{article}
\usepackage[left=2cm,right=2cm,top=1cm,bottom=1cm,bindingoffset=0cm]{geometry}
\usepackage[utf8x]{inputenc}
\usepackage[english,russian]{babel}
\usepackage{cmap}
\usepackage{amssymb}
\usepackage{amsmath}
\usepackage{pifont}
\usepackage{tikz}
\usepackage{verbatim}
\usepackage{enumitem}
\usepackage{hyperref}
\usepackage{float}

\pagenumbering{gobble}

\newcommand{\red}[1]{\textcolor[red]{#1}}

\begin{document}

\begin{center}
{\LARGE Формальные языки}\\

{\Large Самостоятельная работа}\\

{\Large Проничев Андрей}\\

{\Large Вариант 9}
\end{center}

\bigskip

\begin{enumerate}

	\item %1
	$ 	\{a^{3n}b^m \ | \ 1 \le n \le m \le 2n\} $\\
	$ S_0 \to aaaSbB $\\
	$ S \to aaaSbB  \ | \ \varepsilon$\\
	$ B \to b \ | \  \varepsilon $\\
	$ S_0 $ --- стартовый нетерминал.
	
	Понятно, что любая строка будет иметь вид $ a^{\alpha}b^{\beta} $. Более точно, $ n $  применений второго правила даст $ 3n $ символов $ a $ и $ n $ символов $ b $, далее от $ 0 $ до $ n $ нетерминалов $ B $  станут символом $ b $. Итого $ 3n $ символов $ a $ и от $ n $ до $ 2n $ символов $ b $ --- получили искомое множество строк.
	
	\item %2 
	Примеры левостороннего вывода:
	\begin{itemize}
		\item $ S_0 \to a^3SbB \to a^3bB \to a^3b $
		\item $ S_0 \to a^3SbB \to a^3bB \to a^3bb $
		\item $ S_0 \to a^3SbB \to a^6SbBbB \to a^6bBbB \to a^6bbB \to a^6bb $
	\end{itemize}
	Примеры строк, не принадлежащих к языку:
	\begin{itemize}
		\item ababab
		\item aaaaaa
	\end{itemize}
	
	\item %3
	Составим таблицу для $ LL(1) $ анализатора:
	\begin{table}[H]
		\begin{tabular}{l||l|l||l|l|l}
			N    & FIRST    & FOLLOW & a       & b & \$ \\ \hline
			$ S_0  $& \{a\}    & \{\$\} & $ S_0 \to aaaSbB$ &   &    \\ \hline
			$ S  $   & $ \{a, \varepsilon \} $ & $ \{b\} $  &   $ S \to aaaSbB $      & $ S \to \varepsilon $  &    \\ \hline
			$ B $    & $ \{b, \varepsilon\}  $& \{\$, $ b $\} &  & $ B \to b, B \to \varepsilon $  &  $ B \to \varepsilon $  \\
		\end{tabular}
	\end{table}
	Есть конфликт в ячейке $ (B, b) $, а значит грамматика не является $ LL(1) $.
%	Конфликтов нет, можем провести анализ:
%	\begin{itemize}
%		\item Строка $ a^6b^2 $: 
%		\begin{table}[H]
%			\begin{tabular}{ll}
%			Стек& Строка \\
%			\$ $ S_0 $	&  $ aaaaaabb $\\
%			\$ $ B \ b \ S \ a \ a \ a $	&  $ aaaaaabb $\\
%			\$ $ B \ b \ S \ a \ a  $	&  $ aaaaabb $\\
%			\$ $ B \ b \ S \ a   $	&  $ aaaabb $\\
%			\$ $ B \ b \ S    $	&  $ aaabb $\\
%			\$ $ B \ b \ B \ b \ S \ a \ a \ a   $	&  $ aaabb $\\
%			\$ $ B \ b \ B \ b \ S \ a \ a   $	&  $ aabb $\\
%			\$ $ B \ b \ B \ b \ S \ a    $	&  $ abb $\\
%			\$ $ B \ b \ B \ b \ S    $	&  $ bb $\\
%			\$ $ B \ b \ B \ b     $	&  $ bb $\\
%			\$ $ B \ b \ B     $	&  $ b $\\
%			\end{tabular}
%		\end{table}
%	\end{itemize}
	
	\item %4
	Для начала приведём грамматику к НФХ:\\
	$ S $ --- стартовый нетерминал.\\
	$ S \to AD $\\
	$ A \to A_1C_a $\\
	$ A_1 \to C_a C_a $\\
	$ C_a \to a $\\
	$ D \to SB \ | \ C_bC_b \ | \ b $\\
	$ B   \to C_bC_b \ | \ b $\\
	$ C_b \to b $
	
	\begin{itemize}
		\item Рассмотрим анализ цепочки $ a^6b^3 $:\\
	
		\begin{figure}[H]
			\centering
			\includegraphics[width=0.5\linewidth]{2.jpg}
			\caption{Таблица анализатора для строки $ a^6b^3 $}
		\end{figure}
		Полученное дерево вывода:
		\begin{center}
			\begin{tikzpicture}[scale=0.2]
				\tikzstyle{every node}+=[inner sep=0pt]
				\draw [black] (22.3,-2.2) circle (2);
				\draw (22.3,-2.2) node {$S$};
				\draw [black] (16.8,-8.4) circle (2);
				\draw (16.8,-8.4) node {$A$};
				\draw [black] (27,-8.4) circle (2);
				\draw (27,-8.4) node {$D$};
				\draw [black] (34.4,-15.3) circle (2);
				\draw (34.4,-15.3) node {$B$};
				\draw [black] (27.3,-15.3) circle (2);
				\draw (27.3,-15.3) node {$S$};
				\draw [black] (16.8,-15.3) circle (2);
				\draw (16.8,-15.3) node {$C_a$};
				\draw [black] (10,-15.3) circle (2);
				\draw (10,-15.3) node {$A_1$};
				\draw [black] (16.8,-22.8) circle (2);
				\draw (16.8,-22.8) node {$a$};
				\draw [black] (2.2,-22.8) circle (2);
				\draw (2.2,-22.8) node {$C_a$};
				\draw [black] (10,-22.8) circle (2);
				\draw (10,-22.8) node {$C_a$};
				\draw [black] (2.2,-30) circle (2);
				\draw (2.2,-30) node {$a$};
				\draw [black] (10,-30) circle (2);
				\draw (10,-30) node {$a$};
				\draw [black] (34.4,-22.8) circle (2);
				\draw (34.4,-22.8) node {$C_b$};
				\draw [black] (41.5,-22.8) circle (2);
				\draw (41.5,-22.8) node {$C_b$};
				\draw [black] (34.4,-30) circle (2);
				\draw (34.4,-30) node {$b$};
				\draw [black] (41.5,-30) circle (2);
				\draw (41.5,-30) node {$b$};
				\draw [black] (18.9,-30) circle (2);
				\draw (18.9,-30) node {$A$};
				\draw [black] (27.3,-22.8) circle (2);
				\draw (27.3,-22.8) node {$D$};
				\draw [black] (27.3,-30) circle (2);
				\draw (27.3,-30) node {$b$};
				\draw [black] (18.9,-35.5) circle (2);
				\draw (18.9,-35.5) node {$C_a$};
				\draw [black] (13.4,-35.5) circle (2);
				\draw (13.4,-35.5) node {$A_1$};
				\draw [black] (7.2,-42) circle (2);
				\draw (7.2,-42) node {$C_a$};
				\draw [black] (13.4,-42) circle (2);
				\draw (13.4,-42) node {$C_a$};
				\draw [black] (18.9,-42) circle (2);
				\draw (18.9,-42) node {$a$};
				\draw [black] (7.2,-48.4) circle (2);
				\draw (7.2,-48.4) node {$a$};
				\draw [black] (13.4,-48.4) circle (2);
				\draw (13.4,-48.4) node {$a$};
				\draw [black] (20.97,-3.7) -- (18.13,-6.9);
				\fill [black] (18.13,-6.9) -- (19.03,-6.64) -- (18.28,-5.97);
				\draw [black] (23.51,-3.79) -- (25.79,-6.81);
				\fill [black] (25.79,-6.81) -- (25.71,-5.87) -- (24.91,-6.47);
				\draw [black] (16.8,-10.4) -- (16.8,-13.3);
				\fill [black] (16.8,-13.3) -- (17.3,-12.5) -- (16.3,-12.5);
				\draw [black] (27.09,-10.4) -- (27.21,-13.3);
				\fill [black] (27.21,-13.3) -- (27.68,-12.48) -- (26.68,-12.52);
				\draw [black] (28.46,-9.76) -- (32.94,-13.94);
				\fill [black] (32.94,-13.94) -- (32.69,-13.02) -- (32.01,-13.76);
				\draw [black] (15.4,-9.82) -- (11.4,-13.88);
				\fill [black] (11.4,-13.88) -- (12.32,-13.66) -- (11.61,-12.95);
				\draw [black] (16.8,-17.3) -- (16.8,-20.8);
				\fill [black] (16.8,-20.8) -- (17.3,-20) -- (16.3,-20);
				\draw [black] (8.56,-16.69) -- (3.64,-21.41);
				\fill [black] (3.64,-21.41) -- (4.56,-21.22) -- (3.87,-20.5);
				\draw [black] (10,-17.3) -- (10,-20.8);
				\fill [black] (10,-20.8) -- (10.5,-20) -- (9.5,-20);
				\draw [black] (2.2,-24.8) -- (2.2,-28);
				\fill [black] (2.2,-28) -- (2.7,-27.2) -- (1.7,-27.2);
				\draw [black] (10,-24.8) -- (10,-28);
				\fill [black] (10,-28) -- (10.5,-27.2) -- (9.5,-27.2);
				\draw [black] (34.4,-17.3) -- (34.4,-20.8);
				\fill [black] (34.4,-20.8) -- (34.9,-20) -- (33.9,-20);
				\draw [black] (35.77,-16.75) -- (40.13,-21.35);
				\fill [black] (40.13,-21.35) -- (39.94,-20.42) -- (39.21,-21.11);
				\draw [black] (34.4,-24.8) -- (34.4,-28);
				\fill [black] (34.4,-28) -- (34.9,-27.2) -- (33.9,-27.2);
				\draw [black] (41.5,-24.8) -- (41.5,-28);
				\fill [black] (41.5,-28) -- (42,-27.2) -- (41,-27.2);
				\draw [black] (27.3,-17.3) -- (27.3,-20.8);
				\fill [black] (27.3,-20.8) -- (27.8,-20) -- (26.8,-20);
				\draw [black] (26.31,-17.04) -- (19.89,-28.26);
				\fill [black] (19.89,-28.26) -- (20.72,-27.82) -- (19.86,-27.32);
				\draw [black] (27.3,-24.8) -- (27.3,-28);
				\fill [black] (27.3,-28) -- (27.8,-27.2) -- (26.8,-27.2);
				\draw [black] (17.49,-31.41) -- (14.81,-34.09);
				\fill [black] (14.81,-34.09) -- (15.73,-33.87) -- (15.03,-33.17);
				\draw [black] (18.9,-32) -- (18.9,-33.5);
				\fill [black] (18.9,-33.5) -- (19.4,-32.7) -- (18.4,-32.7);
				\draw [black] (18.9,-37.5) -- (18.9,-40);
				\fill [black] (18.9,-40) -- (19.4,-39.2) -- (18.4,-39.2);
				\draw [black] (13.4,-37.5) -- (13.4,-40);
				\fill [black] (13.4,-40) -- (13.9,-39.2) -- (12.9,-39.2);
				\draw [black] (13.4,-44) -- (13.4,-46.4);
				\fill [black] (13.4,-46.4) -- (13.9,-45.6) -- (12.9,-45.6);
				\draw [black] (12.02,-36.95) -- (8.58,-40.55);
				\fill [black] (8.58,-40.55) -- (9.49,-40.32) -- (8.77,-39.63);
				\draw [black] (7.2,-44) -- (7.2,-46.4);
				\fill [black] (7.2,-46.4) -- (7.7,-45.6) -- (6.7,-45.6);
			\end{tikzpicture}
		\end{center}
	
	\item Теперь рассмотрим анализ цепочки $ a^3b^3a $, не принадлежащей нашему языку:
	\begin{figure}[H]
		\centering
		\includegraphics[width=0.5\linewidth]{1.jpg}
		\caption{Таблица анализатора для строки $ a^3b^3a $}
	\end{figure}
	\end{itemize}
	
\end{enumerate}


\end{document}
