\documentclass[a4paper,12pt]{article}
\usepackage{cmap}					% поиск в PDF
\usepackage{mathtext} 				% русские буквы в формулах
\usepackage[T2A]{fontenc}			% кодировка
\usepackage[utf8]{inputenc}			% кодировка исходного текста
\usepackage[english,russian]{babel}	% локализация и переносы
\usepackage{indentfirst}
\frenchspacing
\usepackage[linguistics]{forest}
\usepackage[latin1]{inputenc}
\usetikzlibrary{trees}
\usepackage{amsmath,amsfonts,amssymb,amsthm,mathtools} % AMS
\usepackage{icomma} % "Умная" запятая: 
\DeclareMathOperator{\sgn}{\mathop{sgn}}
\newcommand*{\hm}[1]{#1\nobreak\discretionary{}
{\hbox{$\mathsurround=0pt #1$}}{}}
\usepackage{graphicx}  % Для вставки рисунков
\graphicspath{{images/}{images2/}}  % папки с картинками
\setlength\fboxsep{3pt} % Отступ рамки \fbox{} от рисунка
\setlength\fboxrule{1pt} % Толщина линий рамки \fbox{}
\usepackage{wrapfig} % Обтекание рисунков текстом
\usepackage{array,tabularx,tabulary,booktabs} % Дополнительная работа с таблицами
\usepackage{longtable}  % Длинные таблицы
\usepackage{multirow} % Слияние строк в таблице
\usepackage{xcolor}
\usepackage{hyperref}
\definecolor{linkcolor}{HTML}{799B03} % цвет ссылок
\definecolor{urlcolor}{HTML}{799B03} % цвет гиперссылок
\hypersetup{pdfstartview=FitH,  linkcolor=linkcolor,urlcolor=urlcolor, colorlinks=true}
\theoremstyle{plain} % Это стиль по умолчанию, его можно не переопределять.
\newtheorem{theorem}{Теорема}[section]
\newtheorem{proposition}[theorem]{Утверждение}
\theoremstyle{definition} % "Определение"
\newtheorem{corollary}{Следствие}[theorem]
\newtheorem{problem}{Задача}[section]
\theoremstyle{remark} % "Примечание"
\newtheorem*{nonum}{Решение}
\usepackage{etoolbox} % логические операторы
\usepackage{extsizes} % Возможность сделать 14-й шрифт
\usepackage{geometry} % Простой способ задавать поля
	\geometry{top=25mm}
	\geometry{bottom=35mm}
	\geometry{left=35mm}
	\geometry{right=20mm}
\usepackage{setspace}
\usepackage{lastpage} % Узнать, сколько всего страниц в документе.
\usepackage{soul} % Модификаторы начертания
\usepackage{hyperref}
\usepackage[usenames,dvipsnames,svgnames,table,rgb]{xcolor}
\hypersetup{				% Гиперссылки
    unicode=true,           % русские буквы в раздела PDF
    pdftitle={Заголовок},   % Заголовок
    pdfauthor={Автор},      % Автор
    pdfsubject={Тема},      % Тема
    pdfcreator={Создатель}, % Создатель
    pdfproducer={Производитель}, % Производитель
    pdfkeywords={keyword1} {key2} {key3}, % Ключевые слова
    colorlinks=true,       	% false: ссылки в рамках; true: цветные ссылки
    linkcolor=red,          % внутренние ссылки
    citecolor=black,        % на библиографию
    filecolor=magenta,      % на файлы
    urlcolor=cyan           % на URL
}
\usepackage{csquotes} % Еще инструменты для ссылок
\usepackage{multicol} % Несколько колонок
\usepackage{tikz} % Работа с графикой
\usepackage{pgfplots}
\usepackage{pgfplotstable}

\date{ВШЭ ПМИ, 2022 г.} 
\author{Алексей Косенко}

\begin{document}

% Set the overall layout of the tree
\tikzstyle{level 1}=[level distance=2.5cm, sibling distance=1cm]
\tikzstyle{level 2}=[level distance=2.5cm, sibling distance=1cm]

% Define styles for bags and leafs
\tikzstyle{bag} = [text width=4em, text centered]
\tikzstyle{end} = [circle, minimum width=3pt,fill, inner sep=0pt]

\maketitle

\section{Формальные языки, практика 1.}

\subsection{Привести грамматику для языка палиндромов.}

Язык палиндромов - $\{ w w^r | w \in \{0, 1\}* \}$. Это язык, в котором цепочки начинаются и заканчиваются одинаково. 

$$G = \<\{\{0, 1\}, \{S\}, P, S\}\>$$

Для цепочек четной длины в начало и в конец будем ставить одинаковые терминальные значения, то есть открывающую и закрывающую скобку сразу. 

$$P: S \rightarrow 0S0 \ | \ 1S1 \ | \ \varepsilon$$

Если нам нужна ещё цепочка нечётной длины, то добавим элемент из алфавит с пустым символом, иначе просто $\varepsilon$.

$$P: S \rightarrow 0S0 \ | \ 1S1 \ | \ 1\varepsilon \ | \ 0\varepsilon \ | \ \varepsilon$$


Приведем дерево вывода строки $01100110$ в нашей грамматике, длина которой равна $8$.

\begin{tikzpicture}[grow=right]
\node[bag] {S}
    child {
        node[bag] {0}        
    }
    child {
        node[bag] {S}
            child {
                node[bag] {1}        
            }
            child {
                node[bag] {S}
                    child {
                        node[bag] {1}        
                    }
                    child {
                        node[bag] {S}
                            child {
                                node[bag] {0}        
                            }
                            child {
                                node[bag] {S}
                                    child {
                                        node[bag] {$\varepsilon$}
                                    }
                            }
                            child {
                                node[bag] {0}        
                            }
                    }
                    child {
                        node[bag] {1}        
                    }
            }
            child {
                node[bag] {1}        
            }
    }
    child {
        node[bag] {0}        
    };
\end{tikzpicture}

$$S \rightarrow 0S0 \rightarrow 01S10 \rightarrow 011S110 \rightarrow 0110S0110 \rightarrow 0110\varepsilon0110 \rightarrow 01100110$$
$$$$

\subsection{Описать язык, порождаемый грамматикой.}

$$P: S \rightarrow aSA \ | \ aT$$ 
$$TA \rightarrow bTa$$
$$aA \rightarrow Aa$$
$$T \rightarrow ba$$

\begin{center}
\begin{forest}
\Tree[S  [aT [\textbf{aba}]] [aSA [aaTA [aabTa [\textbf{aabbaa}]]] [aaSAA [aaaSAAA [aaaaTAAA [aaaabTaAA [aaaabTAaA [aaaabbTaaA [aaaabbTaAa [aaaabbTAaa [aaaabbbTaaa [\textbf{aaaabbbbaaaa}]]]]]]]] [aaaaSAAAA [$\dots$]]][aaaTAA [aaabTaA [aaabTAa [aaabbTaa [\textbf{aaabbbaaa}]]]] ]]]]]

\end{forest}
\end{center}

Итак, правила предложенной грамматики описывают цепочки формата: $a^nb^na^n$, где $n \in \mathbb{N}$. 

\subsection{Отрывок из спецификации Python.}

\href{https://docs.python.org/3/reference/index.html}{Вот ссылка на документ}.


\begin{itemize}
    \item В PEP 572 был введён оператор моржа (\textit{walrus operator}) - попытка разграничить оператор присваивания и тоже самое, но в контексте большого выражения. Он не только снижает сложность кода, но и улучшает его чтение. 

    $$a = (1 + (n := 10))$$
    $$print(a, n)$$

    \item Типизация динамически типизированного языка была введена в PEP 483 с помощью стандартной библиотеки $typing$, другими словами - type hinting. Чем-то похоже на TypeScript, но без компилятора, а только с $mypy$. Есть даже доклады, где сравнивается а-ля типизация между JS и Python. Такой "хороший костыль" сделал код быстро читаемым из-за сохранения контекста типов и объектов.

    a: int = 10

    T = TypeVar("T") \# дженерики

    def bar(a: T) $\rightarrow$ T: ...
    
    def foo(a: List[int, ...]) $\rightarrow$ Union[None, Tuple[People]]: ...

    \item Распаковщик списков (\textit{последовательностей}) и словарей. Локаничность и точка.

    a, *b, c = range(10)

    def baz(..., *arg, **kwargs) $\rightarrow$ ... \ :
    
    [*array]; \
    $\{**dict1, **dict2\}$; \
    print(*array)

    \item F-строки, ellipsis, mock-функции (\textit{pass}) и многое другое.
\end{itemize}
\end{document}
